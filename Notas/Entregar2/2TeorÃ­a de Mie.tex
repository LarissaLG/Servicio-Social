\section{Teoría de Mie}
Al estudio de la absorción y el esparcimiento por una esfera de radio e índice de refracción arbitrarios se le conoce como \textbf{Teoría de Mie}. Esta teoría propone la expresión de los campos electromagnéticos que inciden en las partículas como una superposición de armónicos esféricos vectoriales que satisfacen las ecuaciones de Maxwell . \cite{Bohren}\\

Un campo electromagnético armónico ($\Vec{E},\Vec{H}$) en un medio lineal, isotrópico y homogéneo, satisface las ecuaciones de onda vectoriales \cite{Bohren}
\begin{equation}
	\nabla\cdot\Vec{E}+k^2\Vec{E}=0 \hspace{1cm}\nabla\cdot\Vec{H}+k^2\Vec{H}=0,
\end{equation}
donde $k^2=\omega^2\epsilon\mu$ es la relación de dispersión. Estos campos tienen divergencia nula y, suponiendo que no hay fuentes y que el medio es el vacío,
\begin{equation}
	\nabla\times\Vec{E}=i\omega\mu\Vec{H},\hspace{1cm}\nabla\times\Vec{H}=-i\omega\epsilon\Vec{E}.
\end{equation}

Considerando una función escalar $\psi$ y un vector constante arbitrario $\Vec{c}$, se puede construir la función vectorial
\begin{equation}
    \Vec{M}=\nabla\times(\Vec{c}\psi),
    \label{M}
\end{equation}
cuya divergencia es cero y en consecuencia cumple que
\begin{equation*}    
	\nabla\times(\nabla\times \Vec{M})=\nabla(\nabla\cdot \Vec{M})-\nabla^2\Vec{M}=-\nabla^2\Vec{M}= -\nabla\times\Vec{c}(\nabla^2\psi)).
\end{equation*}
Es decir,
\begin{equation*}   
	\nabla^2\Vec{M}=\nabla\times\Vec{c}(\nabla^2\psi).
\end{equation*}
De esta forma, se observa que $\Vec{M}$ satisface la ecuación de onda vectorial si $\psi$ es solución de la ecuación de onda escalar
\begin{equation*}   
	\nabla^2\psi+k^2\psi=0.
\end{equation*}
Considerando otra función vectorial
\begin{equation}
	\Vec{N}=\frac{\nabla\times\Vec{M}}{k},	
\end{equation}
se tiene que por construcción, tiene divergencia igual a cero y satisface la ecuación de onda vectorial. Más aún, 
\begin{equation}
	\nabla\times\Vec{N}=k\Vec{M},
\end{equation}
entonces, 
$$\nabla^2\Vec{N}=\nabla(\nabla\cdot\Vec{N})-(\nabla\times(\nabla\times\Vec{N}))=-\nabla\times(k\Vec{M})=-k(\nabla\times\Vec{M}),$$
y por tanto, 
$$\nabla^2\Vec{N}+k^2\Vec{N}=-k(\nabla\times\Vec{M})+k^2\left(\frac{\nabla\times\Vec{M}}{k}\right)=0.$$
Es así como dado que se tiene que $\Vec{N}$ y $\Vec{M}$ satisfacen la ecuación de onda vectorial, tienen divergencia igual a cero, el rotacional de $\Vec{M}$ es proporcional al de $\Vec{N}$ y viceversa,  pueden ser considerados como campos electromagnéticos, mejor conocidos como \textbf{armónicos esféricos vectoriales}. Debido a la condición de $\Vec{M}$ para satisfacer la ecuación de onda se puede notar que para encontrar soluciones a las ecuaciones de Maxwell se tienen que encontrar soluciones para que $\psi$, mejor conocida como la función generadora de los vectores armónicos $\Vec{M}$ y $\Vec{N}$, satisfaga la ecuación de onda escalar.\\

Considerando a $\Vec{c}$ en la Ec.\ref{M} como el vector radial $\Vec{r}$ ,
$$\Vec{M}=\nabla\times(\Vec{r}\psi)$$
se tiene la solución a la ecuación de onda vectorial en coordenadas esféricas, por lo que es conveniente estudiar la ecuación de onda escalar en coordenadas esféricas dada por
$$\frac{1}{r^2}\frac{\partial}{\partial r}\left(r^2\frac{\partial\psi}{\partial r}\right)+\frac{1}{r^2\sin\theta}\frac{\partial}{\partial\theta}\left(\sin\theta\frac{\partial\psi}{\partial\theta}\right)+\frac{1}{r^2\sin^2\theta}\frac{\partial^2\psi}{\partial\phi^2}+\Vec{k}^2\psi=0.$$
Empleando el método de separación de variables, se obtienen las soluciones
\begin{equation}
    \frac{d^2\Phi}{d\phi^2}+m^2\Phi=0
    \label{Phi}
\end{equation}
\begin{equation}
    \frac{1}{\sin\theta}\frac{d}{d\theta}\left(\sin\theta\frac{d\Theta}{d\theta}\right)+\left[n(n+1)-\frac{m^2}{\sin^2\theta}\right]\Theta=0
    \label{Theta}
\end{equation}
\begin{equation}
    \frac{d}{dr}\left(r^2\frac{dR}{dr}\right)+[k^2r^2-n(n+1)]R=0
    \label{R}
\end{equation}
con $n,m$ las constantes de separación determinadas por condiciones que $\psi$ tiene que satisfacer.

\noindent Para la Ec.(\ref{Phi}), dado que es la ecuación del oscilador armónico, se observa de inmediato que sus soluciones están dadas por
$$\Phi_e=\cos(m\phi)\hspace{0.5cm}\mbox{y}\hspace{0.5cm}\Phi_o=\sin(m\phi),$$
con los subíndices $e,o$ denotando even(par) y odd(impar) y $m$ un entero positivo o igual a cero.

\noindent Para la Ec.(\ref{Theta}), dado que su solución tiene que ser finita en   $\theta=0,\pi$ se consideran a las funciones asociadas de Legendre de primer tipo $P_n^m$ de grado $n$ y orden $m$, donde $n=m+1,...$\\

\noindent Para la Ec. (\ref{R}) realizando el cambio de variable $\rho=kr$ y definiendo la función $Z=R\sqrt{\rho}$, las soluciones linealmente independientes corresponden a las funciones esféricas de Bessel
\begin{align*}
    j_n(\rho)=\sqrt{\frac{\pi}{2\rho}}J_{n+1/2}(\rho), \hspace{1cm}   y_n(\rho)=\sqrt{\frac{\pi}{2\rho}}Y_{n+1/2}(\rho),
\end{align*}
y como cualquier combinación de las soluciones anteriores es solución de la última ecuación, las funciones esféricas de Bessel de tercer tipo (funciones de Hankel) son también soluciones:
\begin{align*}
    h_n^{(1)}(\rho)=j_n(\rho)+iy_n(\rho),\hspace{1cm} h_n^{(2)}(\rho)=j_n(\rho)-iy_n(\rho).
\end{align*}
De esta forma, se tiene que las funciones que satisfacen la ecuación de onda escalar en coordenadas esféricas son
\begin{align*}
    \psi_{emn}=\cos (m\phi)P_n^m(\cos\theta)z_n(kr),\hspace{1cm}
    \psi_{omn}=\sin(m\phi)P_n^m(\cos\theta)z_n(kr),
\end{align*}
con $z_n$ alguna de las funciones esféricas de Bessel $j_n,y_n,h_n^{(1)},h_n^{(2)}$ y por consiguiente, los armónicos esféricos vectoriales generados por $\psi_{emn},\psi_{omn}$ son
$$\Vec{M}_{emn}=\nabla\times(\Vec{r}\psi_{emn}),\hspace{1cm}\Vec{M}_{omn}=\nabla\times(\Vec{r}\psi_{omn})$$
$$\Vec{N}_{emn}=\frac{\nabla\times\Vec{M}_{emn}}{k},\hspace{1cm}\Vec{N}_{omn}=\frac{\nabla\times\Vec{M}_{omn}}{k},$$
que de manera explícita, son expresados como\footnote{Considerando a $\Vec{r}=r\hat{e}_r$ y al rotacional en coordenadas esféricas $$\nabla\times\Vec{v}=\frac{1}{r\sin\theta}\left[\frac{\partial}{\partial\theta} (\sin\theta v_{\phi})-\frac{\partial v_\theta}{\partial\phi}\right]\hat{r}+\frac{1}{r}\left[\frac{1}{\sin\theta}\frac{\partial v_r}{\partial\phi}-\frac{\partial}{\partial r}(rv_\phi)\right]\hat{\theta}+\frac{1}{r}\left[\frac{\partial}{\partial r}(r v_\theta)-\frac{\partial v_r}{\partial\theta}\right]\hat{\phi}$$
} 

\begin{align*}
    \Vec{M}_{emn}   &=\left[-\frac{m}{\sin\theta}P_n^m(\cos\theta)z_n(\rho)\sin (m\phi)\right]\hat{\theta}-\left[\cos(m\phi)z_n(\rho)\frac{d}{d\theta}P_n^m(\cos\theta)\right]\hat{\phi}\\
    \Vec{M}_{omn}  &=\left[\frac{m}{\sin\theta}P_n^m(\cos\theta)z_n(\rho)\cos (m\phi)\right]\hat{\theta}-\left[\sin(m\phi)z_n(\rho)\frac{d}{d\theta}P_n^m(\cos\theta)\right]\hat{\phi}\\
    \Vec{N}_{emn}&=\left[\frac{z_n(\rho)}{\rho}\cos(m\phi)n(n+1)P_n^m(\cos\theta)\right]\hat{r} +\left[\cos(m\phi)\frac{d}{d\theta}P_n^m(\cos\theta)\frac{1}{\rho}\frac{d}{d\rho}(\rho z_n(\rho))\right]\hat{\theta}\\&-\left[m\sin(m\phi)\frac{P_n^m(\cos\theta)}{\sin\theta}\frac{1}{\rho}\frac{d}{d\rho}(\rho z_n(\rho))\right]\hat{\phi}\\
    \Vec{N}_{omn}&=\left[\frac{z_n(\rho)}{\rho}\sin(m\phi)n(n+1)P_n^m(\cos\theta)\right]\hat{r}+\left[\sin(m\phi)\frac{d}{d\theta}P_n^m(\cos\theta)\frac{1}{\rho}\frac{d}{d\rho}(\rho z_n(\rho))\right]\hat{\theta}\\&+\left[m\cos(m\phi)\frac{P_n^m(\cos\theta)}{\sin\theta}\frac{1}{\rho}\frac{d}{d\rho}(\rho z_n(\rho))\right]\hat{\phi}
\end{align*}
A partir de las expresiones anteriores, se puede observar que $\Vec{N}_{omn}$ y $\Vec{N}_{emn}$ no tienen componente radial, ya que cuando $r\gg 1$ esta tiende a cero debido a que se trata de ondas transversales pues además, se está considerando la aproximación de campo lejano. Más aún, se puede decir que cualquier solución a la ecuación de onda puede ser expandida en una serie infinita de $\Vec{M}_{emn},\Vec{M}_{omn},\Vec{N}_{emn},\Vec{N}_{omn}$. Estos son conocidos como \textbf{modos normales electromagnéticos} de una partícula esférica. Hay ciertas condiciones que producen que un modo normal sea excitado o no. En general, el campo esparcido es una superposición de los modos normales, cada uno pesado por un coeficiente apropiado $a_n$ o $b_n$. De manera similar, el campo interno es una superposición de los modos normales, cada uno pesado por un coeficiente apropiado $c_n$ o $d_n$.\cite{Bohren}\\

Para una $n$ dada, se requieren cuatro coeficientes desconocidos $a_n,b_n,c_n$ y $d_n$, por lo que se requieren cuatro ecuaciones independientes, que se obtienen de las condiciones de frontera en $r=a$ con $a$ el radio de la esfera
$$E_{i\theta}+E_{s\theta}=E_{1\theta},\hspace{1cm}E_{i\phi}+E_{s\phi}=E_{1\phi},$$
$$H_{i\theta}+H_{s\theta}=H_{1\theta},\hspace{1cm}H_{i\phi}+H_{s\phi}=H_{1\phi},$$
a partir de las cuales, consideramsustituyendo las expresiones de los campos incidente y esparcido\footnote{Véase con detalle en \cite{Bohren}}, se obtienen expresiones 
\begin{align*}
	a_n&=\frac{\mu m^2 j_n(x)[xj_n(x)]'-\mu_1j_n(x)[mxj_n(mx)]'}{\mu m^2j_n(mx)[xh_n^{(1)}(x)]'-\mu_1 h_n^{(1)}(x)[mxj_n(mx)]'}\\
	b_n&=\frac{\mu_1 j_n(mx)[xj_n(x)]'-\mu j_n(x)[mxj_n(mx)]'}{\mu_1 j_n(mx)[xh_n^{(1)}(x)]'-\mu h_n^{(1)}(x)[mxj_n(mx)]}
\end{align*}
Si para una $n$ en particular la frecuencia es tal que uno de estos denominadores es muy pequeño, su modo normal correspondiente dominará en el campo esparcido. El modo $a_n$ domina si la condición 
\begin{equation}
	\frac{[xh_n^{(1)}]'}{h_n^{(1)}(x)}=\frac{\mu_1[mxj_n(mx)]}{\mu m^2 j_n(mx)}
\end{equation}
se satisface y el modo $b_n$ domina si 
\begin{equation}
	\frac{[xh_n^{(1)}]'}{h_n^{(1)}(x)}=\frac{\mu_1[mxj_n(mx)]}{\mu_1  j_n(mx)}
\end{equation}
se satisface. En general, el campo esparcido es la superposición de los modos. Las frecuencias para las cuales estas condiciones se satisfacen son las llamadas frecuencias naturales de la esfera, son complejas y sus modos asociados son llamados virtuales.

Si las partes imaginarias de estas frecuencias complejas son pequeñas comparadas con las partes reales,

Estos coeficientes pueden ser simplificados introduciendo las funciones Ricatti-Bessel
\begin{equation*}
	\psi_n(\rho)=\rho j_n(\rho),\hspace{1cm} \xi_n(\rho)=\rho h_n^{(1)}(\rho)
\end{equation*}
\begin{align*}
	a_n&=\frac{m\psi_n(mx)\psi_n'(x)-\psi_n(x)\psi_n'(mx)}{m\psi_n(mx)\xi_n'(x)-\xi_n(x)\psi_n'(mx)}\\
	b_n&=\frac{\psi_n(mx)\psi_n'(x)-m\psi_n(x)\psi_n'(mx)}{\psi_n(mx)\xi_n'(x)-m\xi_n(x)\psi_n'(mx)}
\end{align*}

\subsection{Secciones transversales}


