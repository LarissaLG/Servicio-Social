\section{Modelo de Drude}
Este modelo se basa en la diferencia de las propiedades ópticas entre conductores y no conductores y explica cómo es que un metal responde linealmente a un campo eléctrico.


Debido a la periodicidad de una red cristalina, los niveles de energía están agrupados en bandas. Si existe una banda prohibida, the \textit{band gap},entre bandas completamente ocupadas o completamente vacías, el material es un no conductor (aislante o semiconductor. Si por el contrario, una banda no está complemente llena o una banda llena se traspasa con una vacía, el material es un conductor, por lo que los electrones pueden ser excitados a estados no ocupados adyacentes al aplicar un campo eléctrico, produciendo una corriente eléctrica.

En los materiales conductores, dado que existen estados de electrones vacantes en la misma banda de energía hay absorción intrabanda, que es la absorción de fotones de baja energía. Mientras que en los aislantes, cuando hay fotones de mayor energía que el band gap, hay absorción interbanda. De esta forma se puede ver que los aislantes tienden a ser transparentes y reflejan débilmente para fotones con energías menores que el band gap, mientras que los metales tienen a absorber y reflejar en longitudes de onda del visible y del infrarrojo.

