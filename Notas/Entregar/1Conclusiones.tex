\section{Conclusiones}

El objetivo principal de este trabajo fue caracterizar la respuesta óptica de elipsoides de diferentes materiales y tamaños dentro de la nanoescala. Se calcularon las secciones transversales de absorción, esparcimiento y extinción, bajo la aproximación cuasiestática, para partículas elipsoidales oblatas de aluminio, plata, oro, bismuto y óxido de magnesio. Se encontró que para partículas cuyo diámetro es menor a 20 nm, la contribución predominante en la extinción es la absorción, mientras que el esparcimiento es despreciable. Aunado a esto, se encontró que para los materiales que presentan un comportamiento que se ajusta con el modelo de Drude, es posible observar las dos resonancias plasmónicas correspondientes a iluminar la partícula en las direcciones $\hat{e}_x$ y $\hat{e}_z$; mientras que en las partículas semimetálicas, como el oro y el bismuto se observan distintas resonancias, no necesariamente plasmónicas, asociadas a transiciones interbanda o intrabanda y estrechamente relacionadas con sus funciones dieléctricas. Finalmente, para las partículas de óxido de magnesio, se observó que la sección transversal de extinción era creciente, comportamiento debido a su naturaleza dieléctrica.

