\section{Conclusiones}
\begin{itemize}
	
	\item Se caracterizó la respuesta óptica de elipsoides de diferentes materiales y tamaños dentro de la nanoescala por medio de las secciones transversales de absorción, esparcimiento y extinción, bajo la aproximación cuasiestática, para partículas elipsoidales oblatas de aluminio, plata, oro, bismuto y óxido de magnesio. 
	\item Se encontró que para partículas en el límite cuasiestático, la contribución predominante en la extinción es la absorción, mientras que el esparcimiento es despreciable.
	\item En el rango en el que los materiales  presentan un comportamiento que se ajusta con el modelo de Drude, es posible observar las dos resonancias plasmónicas correspondientes a iluminar elipsoides oblatos en las direcciones $\hat{e}_x$ y $\hat{e}_z$.
	\item Las dos resonancias observadas están localizadas a la derecha y a la izquierda de la resonancia correspondiente al iluminar una nanopartícula esférica.
	\item En los rangos en el que la función dieléctrica de los materiales no se ajusta al modelo de Drude el aumento en la sección transversal de extinción promedio se relaciona con contribuciones descritas por el modelo de Lorentz.
	\item Para las partículas de óxido de magnesio, se observó que la sección transversal de extinción es creciente, comportamiento debido a su naturaleza dieléctrica.
	
	
	\end{itemize}

