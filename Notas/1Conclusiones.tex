\section{Conclusiones}
La respuesta óptica de elipsoides de distintos materiales y tamaños dentro de la nanoescala se caracterizó a través del cálculo de las secciones transversales de absorción, esparcimiento y extinción, bajo la aproximación cuasiestática, para nanopartículas elipsoidales oblatas de aluminio, plata, oro, bismuto y óxido de magnesio. Se encontró que, en el límite cuasiestático, la absorción es la contribución predominante en la extinción, mientras que el esparcimiento resulta despreciable. En el rango donde los materiales presentan un comportamiento acorde con el modelo de Drude, se identificaron dos resonancias plasmónicas correspondientes a la iluminación de los elipsoides en las direcciones $\hat{e}_x$ y $\hat{e}_z$, ubicadas hacia el rojo y el azul, respectivamente, de la frecuencia resonancia observada para una nanopartícula esférica. Por otro lado, en los rangos en los que la función dieléctrica de los materiales no se ajusta al modelo de Drude, el incremento en la sección transversal de extinción promedio se asocia con contribuciones descritas por el modelo de Lorentz. Finalmente, para las partículas de óxido de magnesio, se observó que la sección transversal de extinción aumenta con la energía, lo que se atribuye a su naturaleza dieléctrica.
