\section{Conclusiones}
\begin{itemize}
	
	\item Se logró caracterizar la respuesta óptica de elipsoides de diferentes materiales y tamaños dentro de la nanoescala en términos de las secciones transversales de absorción, esparcimiento y extinción, bajo la aproximación cuasiestática, para partículas elipsoidales oblatas de aluminio, plata, oro, bismuto y óxido de magnesio. 
	\item Se encontró que para partículas cuyo diámetro es menor a 5 nm, la contribución predominante en la extinción es la absorción, mientras que el esparcimiento es despreciable.
	\item Em los materiales que presentan un comportamiento que se ajusta con el modelo de Drude, es posible observar las dos resonancias plasmónicas correspondientes a iluminar la partícula en las direcciones $\hat{e}_x$ y $\hat{e}_z$
	\item En las partículas semimetálicas, como el oro y el bismuto se observan distintas resonancias, no necesariamente plasmónicas, asociadas a transiciones interbanda o intrabanda y estrechamente relacionadas con sus funciones dieléctricas
	\item Para las partículas de óxido de magnesio, se observó que la sección transversal de extinción es creciente, comportamiento debido a su naturaleza dieléctrica.
	
	
	\end{itemize}

