\section{Teoría de Mie}
Al estudio de la absorción y el esparcimiento por una esfera de radio e índice de refracción arbitrarios se le conoce como \textbf{Teoría de Mie}. Esta teoría propone la expresión de los campos electromagnéticos que inciden en las partículas como armónicos esféricos vectoriales que satisfacen las ecuaciones de Maxwell y provee una descripción a primer orden de los efectos ópticos en partículas no esféricas.\\

Un campo electromagnético armónico ($\Vec{E},\Vec{H}$) en un medio lineal, isotrópico y homogéneo, satisface las ecuaciones de onda \cite{Bohren}
\begin{equation}
	\nabla\cdot\Vec{E}+k^2\Vec{E}=0 \hspace{1cm}\nabla\cdot\Vec{H}+k^2\Vec{H}=0,
\end{equation}
donde $k^2=\omega^2\epsilon\mu$ es la relación de dispersión. Estos campos tienen divergencia nula
\begin{equation}
	\nabla\cdot\Vec{E}=0,\hspace{1cm}\nabla\cdot\Vec{H}=0,
\end{equation}
suponiendo que no hay fuentes y que el medio es el vacío, y no son independientes
\begin{equation}
	\nabla\times\Vec{E}=i\omega\mu\Vec{H},\hspace{1cm}\nabla\times\Vec{H}=-i\omega\epsilon\Vec{E}.
\end{equation}
Considerando una función escalar $\psi$ y un vector constante arbitrario $\Vec{c}$, a partir de los cuales se construye la función vectorial $\Vec{M}$
\begin{equation}
    \Vec{M}=\nabla\times(\Vec{c}\psi),
\end{equation}
se observa que
\begin{equation}
	\nabla\cdot\Vec{M}=\nabla\cdot(\nabla\times(\Vec{c}\psi))=0.
\end{equation}
Por un lado,
\begin{equation*}    
	\nabla\times(\nabla\times \Vec{M})=\nabla(\nabla\cdot \Vec{M})-\nabla^2\Vec{M}=-\nabla^2\Vec{M};
\end{equation*}
por otro lado, 
\begin{align*}
    \nabla\times(\nabla\times \Vec{M})&=\nabla\times(\nabla\times\Vec{c}\psi)= \nabla\times(\nabla(\nabla\cdot \Vec{c}\psi))-\nabla\times(\nabla^2\Vec{c}\psi))= -\nabla\times\Vec{c}(\nabla^2\psi)).
\end{align*}
Igualando las dos ecuaciones anteriores se tiene
\begin{equation*}   
	\nabla^2\Vec{M}=\nabla\times\Vec{c}(\nabla^2\psi),
\end{equation*}
de esta forma, se obtiene 
\begin{equation*}   
	\nabla^2\Vec{M}+k^2\Vec{M}=\nabla\times[\Vec{c}(\nabla^2\psi+k^2\psi)]
\end{equation*} 
En consecuencia, $\Vec{M}$ satisface la ecuación de onda vectorial si $\psi$ es solución de la ecuación de onda escalar
\begin{equation*}   
	\nabla^2\psi+k^2\psi=0
\end{equation*}
también se puede escribir
$$\Vec{M}=-\Vec{c}\times\nabla\psi$$
donde se puede ver que $\Vec{M}$ es perpendicular a $\Vec{c}$ 
Consideremos otra función vectorial:
$$\Vec{N}=\frac{\nabla\times\Vec{M}}{k}$$
se tiene que por construcción,
$$\nabla\cdot\Vec{N}=0$$
notemos que satisface la ecuación de onda vectorial:
$$\nabla^2\Vec{N}+k^2\Vec{N}=0$$
y además, 
$$\nabla\times\Vec{N}=k\Vec{M}$$
pues 
\begin{align*}
    \nabla\times\Vec{N}&=\nabla\times\left(\frac{\nabla\times\Vec{M}}{k}\right)=\frac{1}{k}[\nabla(\nabla\cdot\Vec{M}-\nabla^2\Vec{M})]=-\frac{1}{k}\nabla^2\Vec{M}=-\frac{1}{k}(-k^2\Vec{M})=k\Vec{M}
\end{align*}
entonces, 
$$\nabla^2\Vec{N}=\nabla(\nabla\cdot\Vec{N})-(\nabla\times(\nabla\times\Vec{N}))=-\nabla\times(k\Vec{M})=-k(\nabla\times\Vec{M})$$
por tanto, 
$$\nabla^2\Vec{N}+k^2\Vec{N}=-k(\nabla\times\Vec{M})+k^2\left(\frac{\nabla\times\Vec{M}}{k}\right)=0$$

Entonces tenemos que $\Vec{N}$ y $\Vec{M}$ satisfacen la ecuación de onda vectorial, tienen divergencia igual a cero, el rotacional de $\Vec{M}$ es proporcional al de $\Vec{N}$ y viceversa, por lo que pueden ser considerados como campos electromagnéticos. Debido a la condición de $\Vec{M}$ para satisfacer la ecuación de onda podemos notar que para encontrar soluciones a las ecuaciones de Maxwell hay que encontrar soluciones para la ecuación de onda escalar. Consideraremos a $\psi$ la función generadora de los vectores armónicos $\Vec{M}$ y $\Vec{N}$ con $\Vec{c}$ el vector guía o piloto.\\

Tomando a $\Vec{r}$ el vector radial:
$$\Vec{M}=\nabla\times(\Vec{r}\psi)$$
es la solución a la ecuación de onda vectorial en coordenadas esféricas. Se puede notar que $\Vec{M}$ es tangencial a cualquier esfera. La ecuación de onda escalar en coordenadas esféricas es:
$$\frac{1}{r^2}\frac{\partial}{\partial r}\left(r^2\frac{\partial\psi}{\partial r}\right)+\frac{1}{r^2\sin\theta}\frac{\partial}{\partial\theta}\left(\sin\theta\frac{\partial\psi}{\partial\theta}\right)+\frac{1}{r^2\sin^2\theta}\frac{\partial^2\psi}{\partial\phi^2}+\Vec{k}^2\psi=0$$

Se utiliza el método de separación de variables buscando soluciones de la forma:
$\psi(r,\theta,\phi)=R(r)\Theta(\theta)\Phi(\phi)$
obtenemos las soluciones:
\begin{equation}
    \frac{d^2\Phi}{d\phi^2}+m^2\Phi=0
\end{equation}
\begin{equation}
    \frac{1}{\sin\theta}\frac{d}{d\theta}\left(\sin\theta\frac{d\Theta}{d\theta}\right)+\left[n(n+1)-\frac{m^2}{\sin^2\theta}\right]\Theta=0
\end{equation}
\begin{equation}
    \frac{d}{dr}\left(r^2\frac{dR}{dr}\right)+[k^2r^2-n(n+1)]R=0
\end{equation}
con $n,m$ las constantes de separación determinadas por condiciones que $\psi$ tiene que satisfacer.\\

Para la primer ecuación notamos que es la ecuación del oscilador armónico, por lo que tenemos que las soluciones son:
$$\Phi_e=\cos(m\phi)\hspace{1cm}\Phi_o=\sin(m\phi)$$
con los subíndices $e,o$ denotando even(par) y odd(impar) y $m$ un entero positivo o igual a cero. Además, necesitamos que $\psi$ sea una función univaluada para el ángulo azimutal $\phi$ para todos los $\phi$ excepto para los puntos en la frontera de las regiones con propiedades diferentes. Aunque en nuestro caso esto no es problema por el problema de la interfaz perfectamente plana.
\begin{equation*}
    \lim_{\nu\rightarrow 2\pi} \psi(\phi+\nu)=\psi(\phi)
\end{equation*}

Como se necesita que las soluciones a la segunda ecuación sean finitas en $\theta=0,\pi$ se consideran a las funciones asociadas de Legendre de primer tipo $P_n^m$ de grado $n$ y orden $m$, donde $n=m+1,...$\\

Estas funciones son ortogonales:

$$\int_{-1}^{1}P_n^m(\mu)P_{n'}^m(\mu)d\mu=\delta_{n'n}\frac{2}{2n+1}\frac{(n+m)!}{(n-m)!}$$

donde $\mu=\cos\theta$. 

Para la última ecuación consideremos $\rho=kr$ una variable adimensional y definimos la función $Z=R\sqrt{\rho}$ entonces
$$\frac{d}{dr}\left(r^2\frac{dR}{dr}\right)+[k^2r^2-n(n+1)]R=0$$
se convierte en 
$$\rho\frac{d}{d\rho}\left(\rho\frac{dZ}{d\rho}\right)+\left[\rho^2-\left(n+\frac{1}{2}\right)^2\right]Z=0$$

tenemos que las soluciones son las funciones de Bessel de primer y segundo tipo $J_v$ y $Y_v$ donde la $v=n+\frac{1}{2}$. Entonces las soluciones linealmente independientes son las funciones esféricas de Bessel:
\begin{align}
    j_n(\rho)&=\sqrt{\frac{\pi}{2\rho}}J_{n+1/2}(\rho)\\
    y_n(\rho)&=\sqrt{\frac{\pi}{2\rho}}Y_{n+1/2}(\rho)
\end{align}

Estas satisfacen las relaciones de recurrencia:
\begin{align}
    z_{n-1}(\rho)+z_{n+1}(\rho)&=\frac{2n+1}{\rho}z_n(\rho)\\
    (2n+1)\frac{d}{d\rho}z_n(\rho)&=nz_{n-1}(\rho)-(n+1)z_{n+1}(\rho)
\end{align}


es decir, los órdenes siguientes los podemos ir generando por recurrencia considerando $z_n=j_n,y_n$. Para los primeros órdenes se tiene que:


como cualquier combinación de las soluciones anteriores es solución de la última ecuación, las funciones esféricas de Bessel de tercer tipo (funciones de Hankel) son también soluciones:
\begin{align}
    h_n^{(1)}(\rho)&=j_n(\rho)+iy_n(\rho)\\
    h_n^{(2)}(\rho)&=j_n(\rho)-iy_n(\rho)\\
\end{align}

De esta forma, tenemos que las funciones que satisfacen la ecuación de onda escalar en coordenadas esféricas son:
\begin{align}
    \psi_{emn}&=\cos (m\phi)P_n^m(\cos\theta)z_n(kr),\\
    \psi_{omn}&=\sin(m\phi)P_n^m(\cos\theta)z_n(kr),\\
\end{align}

con $z_n$ alguna de las funciones esféricas de Bessel $j_n,y_n,h_n^{(1)},h_n^{(2)}$. Cualquier función que satisfaga la ecuación de onda escalar en coordenadas esféricas puede ser expandida como una serie infinita de las funciones $\psi_{emn},\psi_{omn}$ \\

Los armónicos esféricos vectoriales generados por $\psi_{emn},\psi_{omn}$ son:
$$\Vec{M}_{emn}=\nabla\times(\Vec{r}\psi_{emn}),\hspace{1cm}\Vec{M}_{omn}=\nabla\times(\Vec{r}\psi_{omn})$$
$$\Vec{N}_{emn}=\frac{\nabla\times\Vec{M}_{emn}}{k},\hspace{1cm}\Vec{N}_{omn}=\frac{\nabla\times\Vec{M}_{omn}}{k}$$

Sea $\Vec{r}=r\hat{e}_r$ analizando sus componentes y recordando que:
$$\nabla\times\Vec{v}=\frac{1}{r\sin\theta}\left[\frac{\partial}{\partial\theta} (\sin\theta v_{\phi})-\frac{\partial v_\theta}{\partial\phi}\right]\hat{r}+\frac{1}{r}\left[\frac{1}{\sin\theta}\frac{\partial v_r}{\partial\phi}-\frac{\partial}{\partial r}(rv_\phi)\right]\hat{\theta}+\frac{1}{r}\left[\frac{\partial}{\partial r}(r v_\theta)-\frac{\partial v_r}{\partial\theta}\right]\hat{\phi}$$
se tendrá que:
\begin{align*}
    \Vec{M}_{emn}&=r\left\lbrace\frac{1}{r}\left[\frac{1}{\sin\theta}\frac{\partial }{\partial\phi}(\cos (m\phi)P_n^m(\cos\theta)z_n(\rho)))\right]\hat{\theta}+\frac{1}{r}\left[-\frac{\partial }{\partial\theta}(\cos (m\phi)P_n^m(\cos\theta)z_n(kr))\right]\hat{\phi}\right\rbrace\\
    &=\left[-\frac{m}{\sin\theta}P_n^m(\cos\theta)z_n(\rho)\sin (m\phi)\right]\hat{\theta}-\left[\cos(m\phi)z_n(\rho)\frac{d}{d\theta}P_n^m(\cos\theta)\right]\hat{\phi}\\\
\end{align*}

\begin{align*}
    \Vec{M}_{omn}&=r\left\lbrace\frac{1}{r}\left[\frac{1}{\sin\theta}\frac{\partial }{\partial\phi}(\sin (m\phi)P_n^m(\cos\theta)z_n(\rho)))\right]\hat{\theta}+\frac{1}{r}\left[-\frac{\partial }{\partial\theta}(\sin (m\phi)P_n^m(\cos\theta)z_n(kr))\right]\hat{\phi}\right\rbrace\\
    &=\left[\frac{m}{\sin\theta}P_n^m(\cos\theta)z_n(\rho)\cos (m\phi)\right]\hat{\theta}-\left[\sin(m\phi)z_n(\rho)\frac{d}{d\theta}P_n^m(\cos\theta)\right]\hat{\phi}\\\
\end{align*}
\begin{align*}
    \Vec{N}_{emn}&=\frac{1}{kr\sin\theta}\left[-\frac{\partial}{\partial\theta} (\sin\theta \cos(m\phi)z_n(\rho)\frac{d}{d\theta}P_n^m(\cos\theta))-\frac{\partial}{\partial\phi}(-\frac{m}{\sin\theta}P_n^m(\cos\theta)z_n(\rho)\sin (m\phi))\right]\hat{r}+\\
    &+\frac{1}{kr}\left[-\frac{\partial}{\partial r}(-r\cos(m\phi)z_n(\rho)\frac{d}{d\theta}P_n^m(\cos\theta))\right]\hat{\theta}+\frac{1}{kr}\left[\frac{\partial}{\partial r}\left(r (-\frac{m}{\sin\theta}P_n^m(\cos\theta)z_n(\rho)\sin (m\phi))\right)\right]\hat{\phi}\\
    &=\frac{1}{kr\sin\theta}\left[\cos(m\phi)z_n(\rho)\frac{d}{d\theta} \left(-\sin\theta \frac{d}{d\theta}P_n^m(\cos\theta)\right)+\frac{m}{\sin\theta}P_n^m(\cos\theta)z_n(\rho)\frac{d}{d\phi}\left(\sin (m\phi)\right)\right]\hat{r}\\&+\frac{1}{kr}\left[\cos(m\phi)\frac{d}{d\theta}P_n^m(\cos\theta)\frac{d}{d r}(rz_n(kr))\right]\hat{\theta}-\frac{1}{kr}\left[\frac{m}{\sin\theta}P_n^m(\cos\theta)\sin (m\phi)\frac{d}{d r}(r z_n(\rho))\right]\hat{\phi}\\
     &=\frac{1}{kr}\left[\cos(m\phi)z_n(\rho)\frac{1}{\sin\theta}\frac{d}{d\theta} \left(-\sin\theta \frac{d}{d\theta}P_n^m(\cos\theta)\right)+\frac{m^2}{\sin^2\theta}P_n^m(\cos\theta)z_n(\rho)\cos(m\phi)\right]\hat{r}\\
     &+\left[\cos(m\phi)\frac{d}{d\theta}P_n^m(\cos\theta)\frac{1}{\rho}\frac{d}{d r}(rz_n(kr))\right]\hat{\theta}-\left[m\sin(m\phi)\frac{P_n^m(\cos\theta)}{\sin\theta}\frac{1}{\rho}\frac{d}{d r}(rz_n(kr))\right]\hat{\phi}\\
     &=\frac{1}{kr}\left[\cos(m\phi)z_n(\rho)\left\lbrace-\frac{1}{\sin\theta}\frac{d}{d\theta} \left(\sin\theta \frac{d}{d\theta}P_n^m(\cos\theta)\right)+\frac{m^2}{\sin^2\theta}P_n^m(\cos\theta)\right\rbrace\right]\hat{r}\\
     &+\left[\cos(m\phi)\frac{d}{d\theta}P_n^m(\cos\theta)\frac{1}{\rho}\frac{d}{d r}(rz_n(kr))\right]\hat{\theta}-\left[m\sin(m\phi)\frac{P_n^m(\cos\theta)}{\sin\theta}\frac{1}{\rho}\frac{d}{d r}(rz_n(kr))\right]\hat{\phi}\\
\end{align*}

Pero, $P_n^m(\cos\theta)=\Theta(\theta)$ satisface:
\begin{equation*}
    \frac{1}{\sin\theta}\frac{d}{d\theta}\left(\sin\theta\frac{d\Theta}{d\theta}\right)+\left[n(n+1)-\frac{m^2}{\sin^2\theta}\right]\Theta=0
\end{equation*}


$$-\frac{1}{\sin\theta}\frac{d}{d\theta} \left(\sin\theta \frac{d}{d\theta}P_n^m(\cos\theta)\right)+\frac{m^2}{\sin^2\theta}P_n^m(\cos\theta)=n(n+1)P_n^m(\cos\theta)$$

y $kr=\rho$ entonces, 

$$\frac{d}{d r}(rz_n(kr))=\frac{d\rho}{d r}\frac{d}{d\rho}\left(\frac{\rho}{k}z_n(\rho)\right)=\frac{d}{d\rho}(\rho z_n(\rho))$$

de esta forma, 

\begin{align*}
     \Vec{N}_{emn}&=\left[\frac{z_n(\rho)}{\rho}\cos(m\phi)n(n+1)P_n^m(\cos\theta)\right]\hat{r}\\
     &+\left[\cos(m\phi)\frac{d}{d\theta}P_n^m(\cos\theta)\frac{1}{\rho}\frac{d}{d\rho}(\rho z_n(\rho))\right]\hat{\theta}\\&-\left[m\sin(m\phi)\frac{P_n^m(\cos\theta)}{\sin\theta}\frac{1}{\rho}\frac{d}{d\rho}(\rho z_n(\rho))\right]\hat{\phi}
\end{align*}
De manera análoga,
\begin{align*}
     \Vec{N}_{omn}&=\left[\frac{z_n(\rho)}{\rho}\sin(m\phi)n(n+1)P_n^m(\cos\theta)\right]\hat{r}\\
     &+\left[\sin(m\phi)\frac{d}{d\theta}P_n^m(\cos\theta)\frac{1}{\rho}\frac{d}{d\rho}(\rho z_n(\rho))\right]\hat{\theta}\\&+\left[m\cos(m\phi)\frac{P_n^m(\cos\theta)}{\sin\theta}\frac{1}{\rho}\frac{d}{d\rho}(\rho z_n(\rho))\right]\hat{\phi}
\end{align*}

A partir de las expresiones anteriores se puede observar que $\Vec{N}_{omn}$ y $\Vec{N}_{emn}$ no tienen componente radial, ya que cuandi $r\gg 1$ esta tiende a 0 debido a que se trata de ondas transversales pues además, se está considerando la aproximación de campo lejano. Más aún, se puede decir que:
 
\begin{center}
\fbox{\begin{varwidth}{\dimexpr\textwidth-2\fboxsep-2\fboxrule\relax}
Cualquier solución a la ecuación de onda puede ser expandida en una serie infinita de $\Vec{M}_{emn},\Vec{M}_{omn},\Vec{N}_{emn},\Vec{N}_{omn}$
\end{varwidth}}
\end{center}

