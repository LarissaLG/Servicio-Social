\hypertarget{conclusiones}{\section{Conclusiones}}

En este trabajo se estudió la solución analítica del problema de esparcimiento de luz por partículas elipsoidales arbitrarias en la aproximación cuasiestática, analizando las secciones transversales de extinción, absorción y esparcimiento. Además, se estudió el comportamiento del esparcimiento de luz por un elipsoide centrado en el origen dentro de este mismo régimen, comparando los resultados con la respuesta de esferas calculadas también en el límite cuasiestático.\\

Se encontró que, en el régimen cuasiestático, la absorción es la contribución predominante en la extinción para nanopartículas elipsoidales oblatas de aluminio, mientras que el esparcimiento resulta despreciable. En el rango en el que el modelo de Drude se adapta al comportamiento de los materiales, se identificaron dos resonancias plasmónicas desplazadas hacia el rojo y el azul respecto a la frecuencia de resonancia observada en una nanopartícula esférica. Para materiales más realistas, se determinó que es necesario incluir contribuciones adicionales a las plasmónicas, como las descritas por el modelo de Lorentz.

